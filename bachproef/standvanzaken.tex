\chapter{\IfLanguageName{dutch}{Stand van zaken}{State of the art}}
\label{ch:stand-van-zaken}
In het vorige hoofdstuk is de onderzoeksvraag voor dit onderzoek voorgesteld. Ook is de use case van Aucxis kort besproken. In dit hoofdstuk worden eerder gemaakte studies besproken die aansluiten bij de scope van dit onderzoek. Deze literatuur studie moet dienen als vertrek punt voor de volgende hoofdstukken. Er wordt in dit hoofdstuk vooral gedefinieerd wat Continuous Integration \& Continuous Deployment inhoud. Zo kan er een goed beeld worden gevormd van het belang van deze procedures. Ook probeert dit hoofdstuk de volgende hoofdstukken duidelijker te maken met deze studies.
% Tip: Begin elk hoofdstuk met een paragraaf inleiding die beschrijft hoe
% dit hoofdstuk past binnen het geheel van de bachelorproef. Geef in het
% bijzonder aan wat de link is met het vorige en volgende hoofdstuk.

% Pas na deze inleidende paragraaf komt de eerste sectiehoofding.
\section{Continuous Integration \& Continuous Deployment}
Een van de eerste vragen die opdook was: \emph{'Hoe wordt software release management nu eigenlijk verwezenlijkt?'}. \emph{'Zijn er bepaalde procedures of praktijken die worden gevolgd om dit in goede banen te leiden?'}. \emph{'Welke technieken worden er gebruikt?'}. Dit leidde ertoe om in de eerste plaats in de richting van de ITIL-processen te zoeken (ITIL, Information Technology Infrastructure Library). Dit omdat het nauw aansluit bij software release management. Zeker als men denkt in de richting van support doeleinden. 

ITIL is een verzameling van een aantal voorwaarden waaraan moeten worden voldaan in bepaalde situaties. Het zijn de procedures en technieken die een aantal van deze voorwaarden vervullen, die naar Continuous Integration \& Continuous Deployment (CI/CD) hebben geleid. 

\subsection{Software release management voor component gebaseerde software}
De paper \autocite{Hoek2002} is een verslag over een jarenlange observatie van software release management praktijken. Deze paper is niet meer van de jongste maar bespreekt toch nog een aantal belangrijke kernideeën. De technische kant en de gebruikte software is minder relevant.

De paper \autocite{Hoek2002} begint met de problematiek uit te leggen van software release management voor zowel de gebruiker als de ontwikkelaar. In de eerste plaats stelt de paper dat de eindgebruiker altijd het slachtoffer is. Tegenwoordig wordt er veel component gebaseerde software ontwikkelt. Een goed voorbeeld hiervan, zijn Linux packages. Dit maakt het niet altijd gemakkelijk voor de eindgebruiker om de juiste softwarecomponenten te vinden. Laat staan dat het de juiste versies zijn. Voeg daar dan nog aan toe dat sommige softwarebedrijven niet altijd oudere versies aanbieden, dat de eindgebruiker meerdere websites moet afsporen en dat de eindgebruiker ook nog eens verantwoordelijk is voor het installeren en up-to-date houden van al die componenten. Dit is dus ver van een optimale situatie.

Het samenvoegen van verschillende componenten, het uitrollen naar de gebruikers en het updaten ervan, wordt beschreven als software release management. Software release management is enkel verantwoordelijk voor het beheren en het opslaan van de verschillende benodigde componenten en is dus niet bedoelt om de verschillende componenten zelf te compileren. Dit zorgt ervoor dat software release management tools compleet platform onafhankelijk kunnen zijn. Het limiteert daardoor wel de functionaliteit. In het bijzonder zijn de tools niet instaat om aan validatie of versie beheer te doen.

Om aan goede software release management te kunnen doen, is goede documentatie van alle verschillende componenten nodig. De paper stelt een aantal vereisten voor software release management voor. Dit is voor zowel de eindgebruiker als de ontwikkelaar. Deze hebben ze samengesteld uit hun eigen tien jaar lange ervaring. \autocite{Hoek2002}.

\textbf{Minimale vereisten voor ontwikkelaars:}
\begin{itemize}
    \item Afhankelijkheden moeten expliciet zijn en gemakkelijk kunnen worden vastgelegd.
    \item Releases moeten consistent worden gehouden.
    \item De reikwijdte van een release moet controleerbaar zijn.
    \item Het releaseproces zou minimale inspanning aan de kant van de ontwikkelaar moeten inhouden.
    \item Er moet een geschiedenis van opvragen worden bijgehouden.
\end{itemize}
\textbf{Minimale vereisten voor eindgebruiker:}
\begin{itemize}
    \item Beschrijvende informatie moet beschikbaar zijn.
    \item Er moet transparantie over de locatie worden geboden.
    \item Een component en zijn afhankelijkheden moeten als één archief kunnen worden opgehaald.
    \item Software-implementatie hulpmiddelen moeten de software-releasebeheertool kunnen gebruiken als bron voor componenten die moeten worden geïnstalleerd en geconfigureerd.
\end{itemize}
Hierna geeft de paper \autocite{Hoek2002} verder uitleg over een software release management tool die de schrijvers van de paper zelf hebben ontwikkeld. Dit is minder interessant voor de doeleinden van dit onderzoek maar de ideeën achter deze tool kunnen een meerwaarde bieden bij de motivatie waarom software release management zo belangrijk is. 

De bedoeling van deze tool is om enerzijds de informatie die gebruikt wordt in het releasebeheer proces te structureren en anderzijds locatietransparantie aan te bieden.
\begin{figure}[!htbp]
    \centering
    \includegraphics[width=\linewidth]{/Users/kenzie/Documents/HoGent/Bachelorproef/Images/SRM_Concept.png}
    \caption{Figuur uit \autocite{Hoek2002}. Figuur toont een simpel diagram van hoe een software release management systeem opgebouwd kan zijn.}
    \label{fig:SRM_arch}
\end{figure}
De figuur~\ref{fig:SRM_arch} illustreert de architectuur van de software release management tool. Deze bestaat uit vier delen. Een logisch gecentraliseerde, maar fysiek gedistribueerde releasedatabase, een interface waarmee ontwikkelaars componenten in de releasedatabase plaatsen, een interface waarmee gebruikers componenten uit de releasedatabase halen en een webserver voor op afstand toegang te krijgen tot de releasedatabase en de componenten.

Deze paper \autocite{Hoek2002} stelt dus dat software release management ervoor moet zorgen dat componenten die op verschillende locaties, door verschillende bedrijven worden ontwikkeld, gemakkelijk gecentraliseerd toegankelijk moeten zijn. De bedrijven zelf hebben nog altijd volledige controle in handen over het versie beheer en het groeperen van de verschillende extern benodigde componenten. Tevens is de eindgebruiker nog altijd verantwoordelijk voor de installatie ervan maar dit begint ook minder het geval te zijn aangezien er meer release tools op de markt komen.

\subsection{Uitdagingen en problemen in het Release Management Process: een Casestudie}
Zoals eerder aangehaald spelen ITIL-achtige procedures een belangrijke rol in software release management. Deze paper \autocite{Lahtela2011} is een korte casestudie met de bedoeling om kort een aantal problemen aan het licht te stellen in verband met software release binnen een bedrijf of organisatie en hiervoor een oplossing aan te bieden.

De casestudie \autocite{Lahtela2011} beschrijft allereerst een definitie voor release management. “Release management omvat mensen, functies, systemen en activiteiten om software- en hardware versies effectief te plannen, verpakken, bouwen, testen en implementeren in een productie omgeving.”~\textcite{Lahtela2011} Deze definitie sluit goed aan bij wat dit onderzoek tracht te verduidelijken. Ook stelt de studie een belangrijk algemeen probleem. Helaas is in de praktijk bij veel bedrijven nog geen sprake van een implementatie van ITIL, ISO, enz. procedures. Dit omdat dit zeer moeilijk te implementeren is in reeds bestaande processen. Het hebben van zo een procedures is niet alleen zeer belangrijk om kwalitatief goede software op te leveren maar ook voor de supportafdeling van een bedrijf. Meestal zijn dit de mensen die het meeste te maken krijgen met deze procedures. Hierbij moet wel gezorgd worden dat er duidelijk verschil gemaakt wordt tussen het beheren van veranderingen en het release management. Vervolgens beschrijft de studie de vastgestelde problemen en een mogelijk oplossing ervoor. Deze zijn letterlijk overgenomen uit de studie~\textcite{Lahtela2011}.
\begin{itemize}
    \item Er is geen gespecificeerd releasebeheerproces.
    \begin{itemize}
        \item Het releasebeheerproces moet worden beschreven en gestroomlijnd om ervoor te zorgen dat iedereen in de organisatie het proces kent.
    \end{itemize}
    \item De rol van releasemanager is onduidelijk.
    \begin{itemize}
        \item Iemand moet worden genoemd voor de rol van releasemanager. Daarnaast moeten de rollen, toewijzingen en verantwoordelijkheden worden beschreven.
    \end{itemize}
    \item De klant weet niet wat de release bevat.
    \begin{itemize}
        \item Meestal worden niet alle aangebrachte veranderingen beschreven of worden deze te technische beschreven waardoor de klant deze niet verstaat. De boodschap is om deze goed te documenteren op een verstaanbare manier.
    \end{itemize}
     \item De uitgifte distributie snelheid is te hoog.
    \begin{itemize}
        \item De release-vensters, die het tijdstip bepalen waarop de release in de productieomgeving van de klant moet worden geïnstalleerd, moeten worden overeengekomen tussen de serviceprovider en de klant, bijvoorbeeld grote releases maandelijks en kleinere releases wekelijks.
    \end{itemize}
    \item De klant denkt dat de serviceprovider niet alle testgevallen kan testen of inspecteren.
    \begin{itemize}
        \item Het testproces, dat deels wordt gedaan binnen het release managementproces, moet worden ontwikkeld. Het proces moet worden beschreven en aan producttesters worden geleerd. Daarnaast moeten de testresultaten aan de klant worden voorgelegd.
    \end{itemize}
    \item Er moeten meer testomgevingen in het testproces zijn.
    \begin{itemize}
        \item Er moet worden onderzocht of er meer testomgevingen nodig zijn. De ideale situatie is dat er voor elke productieomgeving een identieke testomgeving zou zijn.
    \end{itemize}
    \item Het verandermanagement van testomgevingen is onvoldoende.
    \begin{itemize}
        \item Elke wijziging die in een bepaalde testomgeving wordt aangebracht, moet correct worden gedocumenteerd. Op deze manier zijn alle wijzigingen traceerbaar en up-to-date.
    \end{itemize}
    \item Problemen bij versiebeheer.
    \begin{itemize}
        \item Alle versies van verschillende producten moeten worden gedocumenteerd in een klant specifieke lijst, die de serviceprovider vertelt welke geïnstalleerde productieversies de klant heeft.
    \end{itemize}
    \item De caseorganisatie heeft geen specifieke 'release jury'.
    \begin{itemize}
        \item Er is behoefte aan een specifieke jury die releases inspecteert voordat ze in productie worden genomen.
    \end{itemize}
\end{itemize}
Deze studie \autocite{Lahtela2011} biedt een unieke inkijk op een specifiek geval en biedt oplossingen aan voor bepaalde problemen. Sommige van deze problemen sluiten zeer goed aan bij dit onderzoek. Zo is er nood aan een zeer goede test omgeving zodanig dat er niet nodeloos moet worden gereden tussen klant en bedrijf.

\subsection{Methodes en Systemen voor Software Release Management}
Het volgende document \autocite{Barshefsky2005} is een patent dat een bepaalde problematiek met normale software release management probeert op te lossen. Op zich is dit patent niet zo interessant voor dit onderzoek maar de figuren en ideeën waarvoor het bedrijf in kwestie een patent heeft aangevraagd, kunnen wel helpen in het beter begrijpen van software release management en hoe deze wordt toegepast.

Dit document \autocite{Barshefsky2005} begint met zeer algemeen uit te leggen hoe software release management werkt en wat de verschillende stappen zijn die doorlopen worden. Het begint met versie beheer van de software. Waarna deze in een ontwikkelomgeving wordt gebracht. Hierna zal de software meestal naar een test omgeving gaan vooraleer het in productie wordt geplaatst. Al deze verschillende stappen zijn meestal verschillende mensen, in verschillende grote teams, die niets anders dan hun specifieke stap uitvoeren. \emph{Zie diagram~\ref{fig:FL_fig1}}.
\begin{figure}[!htbp]
    \centering
    \includegraphics{/Users/kenzie/Documents/HoGent/Bachelorproef/Images/SRM_P_Fig1.png}
    \caption{Figuur uit \autocite{Barshefsky2005}. Diagram van een niet geautomatiseerde versie beheer procedure.}
    \label{fig:FL_fig1}
\end{figure}
Het document \autocite{Barshefsky2005} beschrijft echter een aantal problematieken bij het idee uit \emph{figuur~\ref{fig:FL_fig1}}. Zo is dit een zeer manueel proces waarbij verschillende mensen deelnemen. Het is dus zeer gemakkelijk om fouten te maken tijdens een van deze stappen. Bijvoorbeeld een merge conflict. Deze problemen hebben ze proberen verhelpen door middel van computer geassisteerde release. \emph{Zie figuur 2~\textcite{Barshefsky2005}}. Ook is er in dit document nagedacht over de verschillende files en methodes om software te compileren en hoe deze moeten worden beheerd. Zo stellen ze gescheiden opslagplaatsen voor, voor inventory bestanden en compileer bestanden, enz. \emph{Zie figuur 3~\textcite{Barshefsky2005} voor details}.

Dit document \autocite{Barshefsky2005} toont dus een nogal algemeen beeld over de software release cycli en wordt meer ter info beschouwd in dit onderzoek.

\subsection{Continuous Integration en de Tools}
Bij release management hoort continuous integration en continuous deployment (CI/CD). Het is namelijk een toepassing om op een snelle \& kwalitatieve manier software te gaan ontwikkelen en uitrollen. Het volgende korte artikel \autocite{Meyer2014} bespreekt kort waarom een organisatie aan continuous integration moet doen en wat voor tools er allemaal bestaan. Ook worden er een aantal aanwijzingen gegeven rond het gebruik van continuous integration.

Het artikel \autocite{Meyer2014} vertrekt vanuit het standpunt dat een versie beheer tool in ieder softwarebedrijf een gegeven moet zijn. Dit om situaties te vermijden dat software lokaal werkt maar op andere toestellen niet. Ook staat dit toe om volledige geautomatiseerde pipelines te maken die automatisch de code compileert en controleert op fouten door meegeleverde testen uit te voeren of door een uitrol in een test omgeving. Het artikel gaat verder met te stellen dat het bij iedere organisatie een prioriteit moet zijn om ten aller tijden de opgeleverde code werkende te houden. Dit staat dan niet alleen toe dat er ten aller tijden een uitrol kan worden uitgevoerd van werkende code naar een test omgeving, maar ook dat er geen fouten of niet werkende code wordt gepusht door de programmeurs. Het artikel beschrijft verder een aantal tools zoals onder andere 'Jenkins' voor het automatisch uitrollen en testen van nieuwe software.

Dit artikel \autocite{Meyer2014} is minder belangrijk voor dit onderzoek maar toont wel het belang aan van een compileer pipeline en een goeie test omgeving voor code werkende te houden en de kwaliteit te behouden.

\subsection{DevOps}
DevOps is de gebruikte term om de samenwerking te beschrijven tussen softwareteams en hardwareteams binnen een organisatie of project. Deze term wordt meestal gebruikt binnen een release management of CI/CD werk sfeer. Met DevOps is het de bedoeling om de verschillende teams onderling te doen communiceren en samenwerken. Dit met doel om beter en sneller software te ontwikkelen en op te leveren. Het volgende artikel \autocite{Ebert2016} legt in meer detail DevOps uit. De 2 luiken van Devops worden besproken alsook waarom Devops belangrijk is. Ook worden veel tools en technieken aangehaald.

Het artikel \autocite{Ebert2016} begint met uit te leggen wat DevOps voor een organisatie kan betekenen. Het stelt dat Devops staat voor een betere samenwerking tussen ontwikkelen, kwaliteit, zekerheid en operaties. Het is lang niet zo dat DevOps voor ieder softwareproject kan worden gebruikt, maar het wordt toch aangeraden. Ook vertrekt het artikel uit een gegeven dat er al software versiebeheer in enige vorm aanwezig is.

Verder stelt het artikel \autocite{Ebert2016} dat er 2 facetten zijn bij DevOps. Enerzijds de compileer kant van de software en anderzijds het deployment gedeelte van de software. Compileren wordt volgens het artikel hoofdzakelijk op twee manieren uitgevoerd. Aan de ene kant zijn er de traditionele compileer tools die meestal nog wat handmatig werk vereisen. Om in deze tools te compileren moet er meestal een XML-bestand worden gemaakt met specifieke onderdelen. Dit kan nogal arbeidsintensief zijn. Langs de andere kant zijn er continuous integration procedures. Hierbij wordt getracht om op ieder moment testbare, werkende code te hebben. Deze tools zijn een stuk gemakkelijker om te gebruiken en zijn meestal Cloud gebaseerd. Dit omdat het de bedoeling is om op een vlot en sneller tempo updates te kunnen uitrollen. Het artikel beschrijft een tabel~\textcite{Ebert2016} waarin een aantal tools met wat rand info beschreven staan.

Het artikel \autocite{Ebert2016} stelt dat meestal de software moet worden getest hierna, zodanig dat kwaliteit kan worden gegarandeerd. Dit moet zo efficiënt en snel mogelijk gebeuren waardoor er dus nood is aan 'infrastructure as code'. Herbruikbare code die snel kan worden uitgevoerd over meerdere platformen. Verschillende automatisatie tools worden door het artikel besproken. De meeste zijn simpel in gebruik en gebruiken YAML of XML voor de test omgevingen te definiëren. Het uitrollen van test omgevingen voor de software wordt meestal in de Cloud gedaan of met virtuele machines. Dit is afhankelijk van wat de vereisten zijn van de organisatie. 

Ook bespreekt het artikel \autocite{Ebert2016} een van de nieuwere manieren om software te testen. Ze bespreken het gebruik van microservices in de Cloud. Hoofdzakelijk bespreekt het artikel de producten van Amazon AWS. Deze zouden zeer gemakkelijk te automatiseren zijn en gemakkelijk te gebruiken.

Dit artikel \autocite{Ebert2016} is een meerwaarde voor dit onderzoek omdat er een hele hoop tools, naast elkaar, kort worden besproken. De nadruk ligt niet op gebruiksvriendelijkheid maar eerder op wat de tools ondersteunen en of dat ze bruikbaar zijn voor DevOps procedures. Dit artikel bevestigd dat er in de use case van dit onderzoek een nood is aan een goed gebouwde omgeving omdat kwaliteit een nummer één prioriteit geworden is.

\subsection{Continuous Integration en Release Pipelines bouwen door het toepassen DevOps Principes: een Casestudie bij Varidesk}
Zoals de vorige artikelen aantonen is het gebruik van CI/CD binnen software release management aangeraden. Er zijn reeds een aantal problemen en moeilijkheden beschreven. Het volgende artikel \autocite{Debroy2018} is een studie over een specifieke use case van een bedrijf dat bepaalde web services aanbiedt aan klanten. Het beschrijft hoe de ervaring was om een CI/CD pipeline te migreren naar de Cloud. Welke valkuilen er zijn. Welke moeilijkheden er zijn vastgesteld.

De casestudie \autocite{Debroy2018} begint met de nood voor CI/CD uit te leggen. Het beschrijft dat er verschillende voordelen zijn aan CI/CD. Een voordeel is dat er veel sneller kan worden gereageerd op support vragen en mogelijke problemen of updates. Ook beschrijft het artikel dat er in sommige gevallen, zoals bij ‘HP Laserjet Frimware Division’, een kost reductie tot 78 procent is. Vervolgens verklaart de studie dat ondanks CI/CD een wijd verspreide procedure is, er nog weinig onderzoek naar is gebeurd. Ook beschrijft het artikel dat tool ondersteuning niet al te best is.

Vervolgens beschrijft het artikel \autocite{Debroy2018} de oorspronkelijke werkwijze van het bedrijf. Het bedrijf gebruikte voor release en compilatie van hun volledige web service 'Visual Studio Team Service' (VSTS). Dit is een betalend platform van Microsoft. Het platform heeft een marktplaats waarop voor gedefinieerde stappen staan om bepaalde taken te voltooien. Bijvoorbeeld compileren, automatische testen, kwaliteit stappen, enz. Dit voor allerlei soorten projecten en programmeertalen. Ook bestaan er stappen voor automatische uitrol met behulp van bepaalde bestaande tools. Ook bestaat er een mogelijkheid dat de klanten die gebruik maken van het platform, zelf bepaalde taken definiëren en dan uploaden op de marktplaats. 

Het artikel \autocite{Debroy2018} beschrijft dat dit voor de totale website meer dan voldoende is. Al werd er volgens het artikel een aantal beperkingen vastgesteld. Zo is de compilatietijd op dit platform niet schitterend. Het duurt vaak lang. Ook is dit een Cloud specifieke oplossing waardoor dit niet schaalbaar is over verschillende Cloud platformen. Daarboven op is het ook nog eens niet schaalbaar op het platform zelf. Dit heeft het artikel proberen oplossen door meerdere instanties te doen draaien op hetzelfde platform maar de resultaten vielen tegen.

Daarna beschrijft het artikel \autocite{Debroy2018} de gevonden oplossing. Er is een architecturale verandering gebeurd aan de site waardoor er met componenten wordt gewerkt. Dit heeft een nood ontwikkeld voor meerdere pipelines tegelijk. Deze oplossing moet dan ook nog eens redundant zijn door het te verspreiden over verschillende platformen. Een oplossing hiervoor zouden containers kunnen zijn. Dit is een redelijk gemakkelijk te implementeren oplossing. Alleen viel de schaalbaarheid op een hetzelfde platform tegen waardoor de compileertijden niet veel verbeterden. Dus al snel werd er gekeken naar microservices. Dit is gemakkelijk te configureren en extreem schaalbaar op de Cloud platformen. Het enige probleem is het uploaden naar de Cloud is bij ieder platform anders en zeer manueel. Als oplossing hiervoor is een script ontwikkeld dat gemakkelijk aanpasbaar is voor naar de verschillende Cloud platformen te uploaden. Dit heeft, zoals in het artikel aangegeven ~\textcite{Debroy2018}, de compilatietijden drastisch verlaagd.

Deze studie \autocite{Debroy2018} sluit zeer nauw aan bij dit onderzoek. Er wordt beschreven wat mogelijke valkuilen kunnen zijn en waar de problemen liggen. Het toont ook aan dat zo goed als ieder Cloud platform kan worden gebruikt voor CI/CD door het gebruik van containers. Al is dit niet altijd de meest efficiënte oplossing. Ondanks dat dit artikel al een groot deel onderzocht heeft, is er nog altijd nood om te onderzoeken hoe de verschillende platformen CI/CD mogelijk maken op een niet containerized manier.

\section{Amazon AWS}
\subsection{Prestatie Analyse van hoge prestatie reken applicaties op Amazon Web Services Cloud}
Dit onderzoek wil ook de Cloud platformen en hun aanbod naast elkaar leggen voor vergelijking. Ook wilt dit onderzoek de lezer een idee geven over hoe de verschillende datacenters van de Cloud platformen samengesteld zijn, op hardware vlak, en hoe de verbindingen hiermee zijn. De volgende paper \autocite{Jackson2010} is een prestatie test van een Amazon AWS datacenter.

De paper \autocite{Jackson2010} gebruikt synthetische wetenschappelijke testen om de prestatie van een aantal Amazon producten te testen. Een Amazon datacenter is wat raar samengesteld aangezien de eindgebruiker geen enkel idee op voorhand heeft over welke hardware hij krijgt toegewezen in het datacenter (Op het moment van schrijven van \autocite{Jackson2010}). Bovendien is Amazon benadeelt op vlak van connectiviteit omdat het niet directe lijnen heeft naar de datacenters zoals Microsoft met Azure. Belangrijk is dat alle testen uitgevoerd door de paper, zijn uitgevoerd in geografisch hetzelfde datacenter.

In deze paper \autocite{Jackson2010} testen de onderzoekers vooral of dat een Amazon datacenter geschikt zou zijn voor wetenschappelijk gebruik. Vandaar dat er vooral synthetische wetenschappelijke tools worden gebruikt. De tools zijn zorgvuldig geselecteerd zodanig dat de verschillende aspecten van een datacenter worden getest. Zowel de processor, als RAM, de harde schijven en ook de verbinding met de servers worden getest.

De paper \autocite{Jackson2010} concludeert dat de prestatie van de systemen wel goed is maar dat dit afhankelijk is, van welk merk CPU de gebruiker krijgt. Aangezien hier geen controle over is, is het moeilijk om voor dezen redenen Amazon voor rekenintensieve doeleinden aan te raden. Ook is vastgesteld dat de connectie met het datacenter een aanzienlijke beperking is voor taken die zeer transactie intensief zijn.

Deze paper \autocite{Jackson2010} is niet zo zeer een meerwaarde voor dit onderzoek. Het artikel is redelijk verouderd en ondertussen is Amazon een van de grotere Cloud platformen. Ook bespreekt deze paper niet zo goed de producten van Amazon. Ook is de use case die de paper beschrijft totaal verschillend van de use case die dit onderzoek gebruikt. 

\section{Microsoft Azure}
\subsection{Microsoft Azure en Cloud computing}
Aangezien binnen de use case van dit onderzoek Microsoft Azure een grote rol speelt, is het belangrijk om een zeer goed idee te vromen van hoe dit Cloud platform is opgebouwd en wat de verschillende services en producten zijn. Daarom is het volgende hoofdstuk uit het boek \autocite{Copeland2015} redelijk informatief, zeker als een initiatie.

Het eerste hoofdstuk uit het boek \autocite{Copeland2015} is vooral bedoelt als een eerste kennismaking met het Azure platform. Azure is eigenlijk ontstaan uit Microsoft hun office 365 aanbiedingen. Microsoft biedt een grote hoeveelheid aan applicaties aan als onder deel van dit pakket. Omdat die applicaties ergens moeten draaien, is Microsoft begonnen met het bouwen van datacenters om daar hun SaaS (Software as a Service) in te hosten. Al snel beseften ze dat er geld viel te verdienen met het aanbieden van remote services. Het duurde dan ook niet lang vooraleer Microsoft ook begon met IaaS (Infrastructure as a Service) aan te bieden. Dit was toen een unicum volgens het boek \autocite{Copeland2015}, Aangezien tot dan de meeste Cloud platformen ontstaan zijn vanuit het verhuren van overschot aan rekenkracht uit een mengelmoes van toestellen uit een bepaald datacenter. Dit was onder andere een van de conclusies van een verouderd artikel over Amazon aws \autocite{Jackson2010}. Bij Microsoft waren dat speciaal gebouwde datacenters met die services in gedachten. Ook hun PaaS (Platform as a service) wordt kort aangehaald door het boek \autocite{Copeland2015}.

\textbf{Voorbeelden van Azure Iaas:}
\begin{itemize}
    \item Azure virtual machines
    \item Azure virtual networks
    \item Azure virtual networks gateways
    \item Azure storage solutions
    \item ...
\end{itemize}
\textbf{Voorbeelden van Azure Paas:}
\begin{itemize}
    \item Azure SQL database
    \item Azure website
    \item Azure content delivery network
    \item Azure DevOps
    \item ...
\end{itemize}
Verder legt het boek  \autocite{Copeland2015} kort uit wat vanuit Azure wordt gedaan op vlak van privacy en wetgevingen. Dit is minder interessant voor dit onderzoek. Ook gaat het boek kort in op waarom It professionals voor Azure Cloud of een ander Cloud platform zouden moeten kiezen. Zo haalt het boek  \autocite{Copeland2015} aan dat een Cloud platform weinig onderhoud inhoud, minder dan een lokale opstelling. Ook is de eindgebruiker niet verantwoordelijk voor de hardware. Hun datacenter zijn redundant. Dus er is eigenlijk vrij weinig down time. Ook de aangeboden producten zijn vrij compleet en gemakkelijk onderhoudbaar.

Verder geeft het boek  \autocite{Copeland2015} een korte introductie in het Azure web portaal. Deze heeft dit onderzoek voor kennismaking doeleinden doorgelopen. Veel interessants is hier voor dit onderzoek niet uit te melden. Het hoofdstuk is dus een snelle kennismaking met Azure. Dit dient als een basis voor verder onderzoek. Zo gaan we in deze literatuurstudie nog wat dieper in op Azure DevOps en een kleine hands-on. Ook IaaS van Azure wordt nog wat meer uitgediept.

\subsection{Microsoft Azure Documentation}
Zoals eerder aangehaald in deze literatuurstudie, speelt Microsoft Azure een belangrijke rol binnen de use case van dit onderzoek. Het is het vertrek punt voor de vergelijkingen. In het kader van dit onderzoek, is de website van Azure eens uitgeplozen. Dit omdat er een beeld kan gevormd worden over welke Azure services interessant kunnen zijn. Het volgende is een kort verslag. Er worden twee service categorieën aangehaald. Deze zijn Iaas en PaaS (Infrastructure as a Services en Platform as a Service). Specifiek is er gekeken naar Azure virtual machine, Azure virtual networks, Azure virtual gateways en Azure DeVops. Het laatste is een samenvatting van een onlinecursus en informatie op Microsoft Docs.

In het kader van software release management en dan vooral de stap om de kwaliteit van software te controleren, is er gekeken of er misschien een mogelijkheid is om deze specifieke infrastructuur eventueel in de Cloud te maken. Hiervoor is er een kort concept uitgewerkt. Dit concept beschrijft een hybride infrastructuur waarbij eigenlijk alle niet use case specifieke toestellen in de Cloud zitten. In theorie stond dit toe dat alle infrastructuur dan als code zou kunnen worden gedefinieerd.

Het aanbod qua mogelijkheden voor hardware om een virtuele machine aan te maken op Azure is enorm. Ook in tegenstelling tot de concurrenten wordt er zeer transparant omgesprongen met welke hardware er per optie beschikbaar wordt gesteld. De mogelijkheden verschillen enorm en zijn meestal voor specifieke doeleindes. Ook de prijzen schalen mede met de use cases. Zo zijn er specifieke opties voor machine learning. Of een optie voor machines met enorme hoeveelheden ram en CPU-kracht om met bepaalde databases overweg te kunnen. Ook de goedkopere basis opties zijn redelijk uitgebreid. Al deze opties worden door Azure onderverdeelt in categorieën die door middel van een letter worden aangeduid. Zie figuur~\ref{fig:Chart_Azure_tiers}.
\begin{figure}[!htbp]
    \centering
    \includegraphics[width=\linewidth]{/Users/kenzie/Documents/HoGent/Bachelorproef/Images/azure-vm-types-comparison-1.jpg}
    \caption{Figuur van (https://p2zk82o7hr3yb6ge7gzxx4ki-wpengine.netdna-ssl.com/wp-content/uploads/azure-vm-types-comparison-1.jpg). Chart met alle VM tiers van azure en hun specifieke code.}
    \label{fig:Chart_Azure_tiers}
\end{figure}
Verder is Azure de oudere manier van het definiëren van een Azure virtual machine aan het afraden. Hiernaast zijn er ook nog een tal van mogelijkheden op vlak van virtual networking. Het belangrijkste is dat er een mogelijkheid bestaat om een virtueel netwerk volledig te isoleren van het internet. Dit staat toe dat enkel verkeer dat op dat specifieke netwerk is, aan de virtuele machines kan. Er wordt dan wel meestal een Azure network firewall node voorzien die in dit virtueel netwerk zit, zodanig dat de connectiviteit met de machines ten aller tijden kan worden gegarandeerd. Om dan connectiviteit te hebben met het virtueel netwerk wordt er gebruikgemaakt van een Azure Gateway node. Deze staat een point to point VPN (Virtual Private Network) tunnel toe. Dit is eigenlijk een directe, geëncrypteerde verbinding met het Azure netwerk. Hiervoor is wel specifieke hardware nodig lokaal in het netwerk. De kost van deze services is eigenlijk bijna niks. Azure werkt immers met een pay-to-run, pay-on-the-go systeem. Dit wil zeggen dat er dus moet worden betaald voor de aanmaak van de services en daarna voor het verbruik. Microsoft heeft op zijn documentatie portaal tal van stap voor stap handleiding voor het instellen van deze services. Ook concepten voor hybride opstellingen staan hier uitgelegd. Maar dit zou deze literatuurstudie te veel doen afwijken.

Het zou dus in theorie mogelijk moeten zijn om een hybride Cloud infrastructuur te voorzien die dynamisch is voor de specifieke te testen projecten. De vraag is of dit handig is in gebruik en niet nodeloos complexiteit toevoegt.

Aure DevOps id een platform van Azure dat organisaties toestaat om bepaalde procedures te definiëren voor het uitrollen van projecten. Het platform staat dan ook integratie toe met andere project follow-up tools van Microsoft. Deze procedures kunnen juist zoals de virtuele machines op Azure via code worden gedefinieerd of via een duidelijk en gemakkelijk te gebruiken GUI. De bedoeling van Azure DeVops is eigenlijk om het oude TFS-systeem te vervangen en een verbeterde interface aan te bieden. Meestal wordt DeVops gebruikt om aan Continuous Integration te doen. Dus er wordt een repository waar de code opstaat gedefinieerd. Hierna bestaat er een mogelijkheid om de code te compileren, automatisch testen te runnen en deze code dan weer door te schuiven naar een volgend stadium. Hier wordt de code dan uitgerold naar een omgeving voor verder kwaliteitscontrole. Het mooie aan dit platform is dat de gebruiker of organisatie niet verplicht is om deze code in een virtuele machine in de Cloud te compileren of testen. Er is volledige controle over de procedures.

In het kader van dit onderzoek is dit zeer belangrijk aangezien dit platform op dit moment in gebruik genomen wordt. Er is ook een onlinecursus gevolgd met een basis uitleg en een labo. Dit heeft duidelijk gemaakt dat het mogelijk is om vanuit DevOps op een lokale test omgeving uit te rollen.

%Aantal woorden: 4342%