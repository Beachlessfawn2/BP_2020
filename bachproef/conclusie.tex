%%=============================================================================
%% Conclusie
%%=============================================================================

\chapter{Conclusie}
\label{ch:conclusie}
Er bestaan veel Cloud platformen en nog meer services en producten. Dit onderzoek heeft duidelijk aangetoond dat niet ieder platform even geschikt is om CI/CD op te implementeren. Al vroeg in het onderzoek zijn IBM Cloud, Oracle Cloud en Digital Ocean buiten beschouwing gevallen. Deze Cloud platformen bevatten misschien wel de mogelijkheid om een pipeline te configureren. Maar dit werd meestal gedaan door middel van het Tekton framework. Dit framework maakt het mogelijk dat ieder Cloud platform dat Kubernetes ondersteund, kan worden gebruikt om CI/CD te implementeren. Alleen zijn er dan geen dashboards beschikbaar met specifieke informatie over de resultaten van de pipeline. Het zou interessant zijn om in een later onderzoek dit framework eens te bekijken.

Van de twee alternatieve kandidaten die overbleven was Google Cloud veruit de meest geschikte kandidaat. Dit platform is in het begin moeilijk in gebruik. Eenmaal de eerste pipeline is geconfigureerd, wordt het platform makkelijker in gebruik. Door het gebruik van Docker containers kan Google Cloud gemakkelijk worden aangepast naar de gebruiker zijn noden. Ook zorgen de containers ervoor dat het configureren van een pipeline intuïtief is. De documentatie die ter beschikking is van Google, is duidelijk en gemakkelijk te volgen. Er bestaan genoeg voorbeelden om als basis te gebruiken voor een opstelling. Ook is gebleken uit de gebruiksvriendelijkheid test dat Google Cloud in vergelijking met Amazon AWS gemakkelijker te configureren is. \emph{Tabel~\ref{tab:gebruiksvriendelijkheid}} toont de gespendeerde tijd voor de Cloud platformen. Google Cloud is een goed alternatief is voor Azure DevOps. De analytics zijn minder goed in vergelijking met Azure DevOps. Deze zijn ook minder visueel. Een toekomstig onderzoek kan onderzoeken of er systemen bestaan om betere visuele dashboards te maken. Er kan bijvoorbeeld worden onderzocht of Grafana een oplossing kan zijn.

Ook is gebleken uit de gebruiksvriendelijkheid test dat Amazon AWS slachtoffer is van zijn eigen succes. Er bestaan te veel producten en services waardoor het moeilijk is om de juiste producten te selecteren. Ook de documentatie kan worden verbeterd. Een beginner zal echt problemen hebben bij het opstellen van zijn eerste pipeline door deze documentatie. Een plus punt is de online console. Deze console is even gemakkelijk in gebruik als de console van Azure DevOps. Dit is iets wat Google Cloud in de toekomst zou kunnen verbeteren.

De ‘Proof Of Concept’ op Google Cloud is geslaagd. Er is succesvol een pipeline opgesteld die automatisch wordt uitgevoerd bij het updaten van de broncode. Ook wordt de applicatie geüpload naar een lokale omgeving. Deze is uitvoerbaar en kan getest worden in een lokale omgeving.

Dit onderzoek heeft aangetoond dat binnen het ecosysteem van Microsoft, Azure DevOps de beste keuze is. Dit platform heeft meer mogelijkheden voor CI/CD dan Google Cloud. Ook heeft het meer hulpmiddelen om projecten te begeleiden. Er is geen nood meer om nog andere hulpmiddelen van derden te gebruiken. Het nadeel van Azure DevOps is dat niet al deze hulpmiddelen beschikbaar zijn bij een basis Azure DevOps licentie. Het kost meer om al deze hulpmiddelen ter beschikking te hebben. Daarom is Google Cloud het beste alternatief voor gebruikers die niet binnen dit ecosysteem zitten. Google Cloud evenaart goed de functionaliteit van Azure DevOps. Het platform is zelfs beter aan te passen naar specifieke noden. Beide platformen hebben ook goede ondersteuning voor andere hulpmiddelen voor bijvoorbeeld het uitrollen van de applicatie naar mobiele systemen.

%Trek een duidelijke conclusie, in de vorm van een antwoord op de
% onderzoeksvra(a)g(en). Wat was jouw bijdrage aan het onderzoeksdomein en
% hoe biedt dit meerwaarde aan het vakgebied/doelgroep? 
% Reflecteer kritisch over het resultaat. In Engelse teksten wordt deze sectie
% ``Discussion'' genoemd. Had je deze uitkomst verwacht? Zijn er zaken die nog
% niet duidelijk zijn?
% Heeft het onderzoek geleid tot nieuwe vragen die uitnodigen tot verder 
%onderzoek?

