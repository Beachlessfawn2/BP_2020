%%=============================================================================
%% Inleiding
%%=============================================================================

\chapter{\IfLanguageName{dutch}{Inleiding}{Introduction}}
\label{ch:inleiding}
Heden ten dage zijn er veel bedrijven die grote delen van hun server infrastructuur digitaliseren in de Cloud. Dit kan variëren van eenvoudige webapplicaties tot een volledige workflow. Deze situaties zijn vaak zeer use case specifiek. Daarbovenop helpt het ook niet dat er verschillende Cloud aanbieders zijn met elk hun specifieke serie van producten en oplossingen. Deze werken dan ook nog eens met verschillende prijzenstelsels. Daarbovenop verschillen de Cloud aanbieders ook in functionaliteit. Het kan dus vaak zeer complex zijn om een gepaste oplossing te vinden.

Een bepaalde workflow die belangrijk is binnen een softwarebedrijf, is continuous integration \& continuous deployment (CI/CD). Deze moet ervoor zorgen dat elk stukje software dat door een programmeur of een team van programmeurs wordt opgeleverd, voldoet aan een aantal opgestelde kwaliteit eisen en uiteindelijk bij de klanten in een productieomgeving uitgerold kan worden. CI/CD automatiseert dit proces volledig. 

CI/CD is ondertussen bij veel bedrijven geïmplementeerd. Desondanks is er maar weinig info te vinden over welke platformen of oplossingen nu specifiek de beste zijn. Als er al informatie of onderzoeken over bestaan, dan zijn deze meestal zeer specifiek en enkel in dat geval toe te passen. Daarom is het interessant om voor deze use case te bekijken wat er het best past. Wat is de beste strategie om deze pijpleiding te implementeren.

Om deze automatisatie te bereiken zijn er een aantal strategieën. Een daarvan is met containers werken. Dit zorgt ervoor dat de pijpleiding flexibel en herhaaldelijk is op zo goed als elk Cloud platform dat 'Platform as a Service' (Paas) aanbiedt. Ook zorgt het ervoor dat er containers op maat gemaakt kunnen worden om optimaal van de functies van een bepaald platform gebruik te maken. Dit onderzoek zal op maat gemaakte compileer en test containers buiten beschouwing houden. Dit omdat deze oplossing vaak arbeidsintensief is om te ontwikkelen en minder gebruiksvriendelijk. Dit onderzoek zal dus zo veel mogelijk gebruikmaken van ingebakken en voor gemaakte oplossingen.

%%De inleiding moet de lezer net genoeg informatie verschaffen om het onderwerp te begrijpen en in te zien waarom de onderzoeksvraag de moeite waard is om te onderzoeken. In de inleiding ga je literatuurverwijzingen beperken, zodat de tekst vlot leesbaar blijft. Je kan de inleiding verder onderverdelen in secties als dit de tekst verduidelijkt. Zaken die aan bod kunnen komen in de inleiding~\autocite{Pollefliet2011}:%%

%%\begin{itemize}%%
  %%\item context, achtergrond
  %%\item afbakenen van het onderwerp
   %%\item verantwoording van het onderwerp, methodologie
   %%\item probleemstelling
   %%\item onderzoeksdoelstelling
   %%\item onderzoeksvraag
   %%\item \ldots
%%\end{itemize}%%

\section{\IfLanguageName{dutch}{Probleemstelling}{Problem Statement}}
\label{sec:probleemstelling}
CI/CD Procedures zijn geen nieuw idee. Beschrijving over wat deze procedures zijn en hoe ze het best worden opgesteld zijn goed uitgediept. CI/CD wordt gebruikt om snel kwalitatieve software op te leveren en uit te rollen naar productie omgevingen bij klanten. Ook Aucxis wil zich meer toeleggen op betere en kwalitatievere software, zodanig dat er sneller kwalitatieve support aan de klanten aangeboden kan worden. 

Echter, hoe deze nu het best worden geïmplementeerd op zowel praktisch als technisch vlak, bestaat er nog geen eenduidig algemeen antwoord. De meeste recente onderzoeken beschrijven vaak specifieke gevallen en gebruiken vaak compleet op maat gemaakte oplossingen per Cloud platform. Dit is ook niet zo onlogisch. Onderzoeken over welk Cloud platform optimaal is voor CI/CD zijn vrijwel niet te vinden. Ook zijn er weinig tot geen onderzoeken te vinden over de vergelijking van verschillende mogelijkheden van Cloud platformen. Er bestaan wel talloze onderzoeken over de prestatie van de verschillende platformen. Zo een soort onderzoeken zullen ook in acht worden genomen in dit onderzoek.

Deze paper tracht een duidelijk beeld te vormen voor Aucxis over welke mogelijkheden er nu bestaan, welk platform specifieke CI/CD mogelijkheden heeft en wat er nu het meest gebruiksvriendelijk lijkt.

%%Uit je probleemstelling moet duidelijk zijn dat je onderzoek een meerwaarde heeft voor een concrete doelgroep. De doelgroep moet goed gedefinieerd en afgelijnd zijn. Doelgroepen als ``bedrijven,'' ``KMO's,'' systeembeheerders, enz.~zijn nog te vaag. Als je een lijstje kan maken van de personen/organisaties die een meerwaarde zullen vinden in deze bachelorproef (dit is eigenlijk je steekproefkader), dan is dat een indicatie dat de doelgroep goed gedefinieerd is. Dit kan een enkel bedrijf zijn of zelfs één persoon (je co-promotor/opdrachtgever).%%
\section{Use Case}
\label{sec:usecase}
Aucxis werkt momenteel veel met ‘.NET’ projecten. Als IDE wordt er Visual Studio gebruikt. Als versiebeheer gebruiken ze hiervoor 'Team Foundation Server' (TFS). Dit was een voor een lange tijd een goede oplossing. De testen die worden uitgevoerd zijn vooral functionele testen. Dit is niet ideaal. Er worden veel fouten en bugs vastgesteld op moment van implementatie bij klanten. Dit kan leiden tot veel frustraties bij de mensen verantwoordelijk voor het implementeren van de software op locatie. Anderzijds gaat er veel tijd en geld verloren met het verplaatsen tussen klant en bedrijf. Aucxis heeft recent de keuze gemaakt om meer te investeren in kwaliteit. Aangezien Aucxis al een Microsoft partner is, was de keuze voor Azure DeVops niet moeilijk. Ook heeft dit minder impact op hun huidige werkwijze.

Zoals eerder aangehaald worden er vooral functionele testen uitgevoerd. Met de stap naar kwalitatievere oplossingen is er ook nood voor meerdere soorten testen. Het ideale zou zijn dat er naast functionele testen ook in een gecontroleerde omgeving getest kan worden. Daarna zou het programma bij een test omgeving van de klant uitgerold worden om daar getest te worden. In een finale stap zou het programma dan in productie uitgerold worden. Dit alles zou zo geautomatiseerd mogelijk moeten verlopen.

Azure DeVops lijkt hier het meest geschikt om deze functionaliteit te verkrijgen. Aucxis is reeds aan de overstap naar Azure Devops begonnen. Hierbij is niet echt stilgestaan bij andere mogelijkheden. Daarom de vraag om toch nog het aanbod van Azure met de andere Cloud platform aanbieders te vergelijken en het beste alternatief te selecteren.

\section{\IfLanguageName{dutch}{Onderzoeksvraag}{Research question}}
\label{sec:onderzoeksvraag}
Welk Cloud platform heeft het meest geschikte aanbod voor CI/CD te implementeren, vertrekkend vanuit een specifieke use case van Aucxis, naast het Azure Devops platform. Dit zonder arbeidsintensieve containers te maken voor ieder specifiek platform. Zo zal in dit onderzoek het Tekton framework buiten beschouwing gehouden worden. Dit framework staat immers toe dat ieder Cloud platform een geschikte kandidaat zo zijn. De criteria om een vergelijking te kunnen maken zijn: aanbod, gebruiksvriendelijkheid, haalbaarheid en prijs.

%%Wees zo concreet mogelijk bij het formuleren van je onderzoeksvraag. Een onderzoeksvraag is trouwens iets waar nog niemand op dit moment een antwoord heeft (voor zover je kan nagaan). Het opzoeken van bestaande informatie (bv. ``welke tools bestaan er voor deze toepassing?'') is dus geen onderzoeksvraag. Je kan de onderzoeksvraag verder specifiëren in deelvragen. Bv.~als je onderzoek gaat over performantiemetingen, dan%%

\section{\IfLanguageName{dutch}{Onderzoeksdoelstelling}{Research objective}}
\label{sec:onderzoeksdoelstelling}
Welk Cloud platform nu mogelijk een beter alternatief is voor CI/CD zal worden aangetoond in twee stadia. In een eerste fase worden alle kandidaten naast elkaar gelegd voor vergelijking. Er zal worden gekeken naar hun mogelijkheden, hoe de prijzenstelsels werken, eerdere implementaties van andere toepassingen en de gebruiksvriendelijkheid. Daaruit wordt dan het beste alternatief voor Azure Devops gekozen. In de tweede fase zal met Azure Devops en het beste alternatief een Proof Of Concept uitgewerkt worden. Voor deze Proof Of Concept wordt er een pijpleiding voor een Microsoft specifieke applicatie opgesteld. Deze pijpleiding moet kunnen automatisch geactiveerd worden. Ook moet deze pijpleiding een gecompileerde applicatie uploaden naar een lokale test omgeving. Deze twee platformen worden dan enigszins met elkaar vergeleken. Dit alles zou een vergelijkend beeld moeten vormen over welk platform het meest geschikt is om met bestaande functies een pijpleiding te implementeren.

%%Wat is het beoogde resultaat van je bachelorproef? Wat zijn de criteria voor succes? Beschrijf die zo concreet mogelijk. Gaat het bv. om een proof-of-concept, een prototype, een verslag met aanbevelingen, een vergelijkende studie, enz.%%

\section{\IfLanguageName{dutch}{Opzet van deze bachelorproef}{Structure of this bachelor thesis}}
\label{sec:opzet-bachelorproef}

% Het is gebruikelijk aan het einde van de inleiding een overzicht te
% geven van de opbouw van de rest van de tekst. Deze sectie bevat al een aanzet
% die je kan aanvullen/aanpassen in functie van je eigen tekst.

De rest van deze bachelorproef is als volgt opgebouwd:

In Hoofdstuk~\ref{ch:stand-van-zaken} wordt een overzicht gegeven van de stand van zaken binnen het onderzoeksdomein, op basis van een literatuurstudie.

In Sectie~\ref{sec:Kandidaten} worden de Cloud platformen hun aanbod kort besproken en vergeleken.

In Hoofdstuk~\ref{ch:methodologie} wordt de methodologie toegelicht en worden de gebruikte onderzoekstechnieken besproken om een antwoord te kunnen formuleren op de onderzoeksvragen.

In Sectie~\ref{sec:KandidaatSelectie} wordt besproken hoe gebruiksvriendelijk de Cloud platformen zijn aan de hand van een simpele opstelling.

in Sectie~\ref{sec:POC} wordt met de geselecteerde kandidaat uit voorgaande sectie, een Proof Of Concept opgesteld voor een Microsoft gerelateerde use case.

In Hoofdstuk~\ref{ch:conclusie}, tenslotte, wordt de conclusie gegeven en een antwoord geformuleerd op de onderzoeksvragen. Daarbij wordt ook een aanzet gegeven voor toekomstig onderzoek binnen dit domein.

%Aantal woorden: 727%