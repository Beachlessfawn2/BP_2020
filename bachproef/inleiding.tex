%%=============================================================================
%% Inleiding
%%=============================================================================

\chapter{\IfLanguageName{dutch}{Inleiding}{Introduction}}
\label{ch:inleiding}
Heden ten dage zijn er veel bedrijven die grote delen van hun server infrastructuur digitaliseren in de Cloud. Dit kan variëren van simpele webapplicaties tot een volledige workflow. Deze situaties zijn vaak zeer use case specifiek. Daarbovenop helpt het ook niet dat er verschillende cloud aanbieders zijn met elk hun specifieke serie van producten en oplossingen die dan ook nog eens met verschillende prijzen stelsels werken en onderling, tussen de verschillende aanbieders, verschillen in functionaliteit. Het kan dus vaak zeer complex zijn om een gepaste oplossing te vinden.

Een bepaalde workflow die zeer belangrijk is binnen een softwarebedrijf, is de continuous development \& continuous deployment (CI/CD) pijpleiding. Dit is een zeer essentiële pijpleiding. Deze moet ervoor zorgen dat elk stukje software dat door een programmeur of een team van programmeurs wordt opgeleverd, voldoet aan een aantal opgestelde kwaliteit eisen en uiteindelijk bij de klanten in een productie omgeving uitgerold kan worden. De CI/CD pijpleiding automatiseert dit proces volledig. 

CI/CD is ondertussen bij veel bedrijven geïmplementeerd. Des ondanks is er maar weinig info te vinden over welke platformen of oplossingen nu specifiek de beste zijn. Als er dan informatie of onderzoeken over bestaan dan is deze meestal zeer specifiek en enkel in dat speciaal geval toe te passen. (verwijzing) Daarom is het nog interessant om in deze use case te bekijken wat er het best past. Wat is de beste strategie om deze pijpleiding te gaan implementeren.

Om deze automatisatie te bereiken zijn er een aantal strategieën. Een daarvan is met containers werken. Dit zorgt ervoor dat de pijpleiding flexibel en herhaalbaar is op zo goed als elk cloud platform dat Platfrom as a Service (Paas) aanbiedt. Ook zorgt het ervoor dat er containers op maat kunnen gemaakt worden om optimaal van een bepaald platform zijn functies gebruik te maken. Deze paper zal op maat gemaakte compile en test containers buiten beschouwing houden. Dit omdat deze oplossing vaak arbeidsintensief is om te ontwikkelen en minder gebruiksvriendelijk is. Deze paper zal dus zo veel mogelijk gebruikmaken van ingebakken en voor gemaakte oplossingen voor de vergelijkingen.

%%De inleiding moet de lezer net genoeg informatie verschaffen om het onderwerp te begrijpen en in te zien waarom de onderzoeksvraag de moeite waard is om te onderzoeken. In de inleiding ga je literatuurverwijzingen beperken, zodat de tekst vlot leesbaar blijft. Je kan de inleiding verder onderverdelen in secties als dit de tekst verduidelijkt. Zaken die aan bod kunnen komen in de inleiding~\autocite{Pollefliet2011}:%%

%%\begin{itemize}%%
  %%\item context, achtergrond
  %%\item afbakenen van het onderwerp
   %%\item verantwoording van het onderwerp, methodologie
   %%\item probleemstelling
   %%\item onderzoeksdoelstelling
   %%\item onderzoeksvraag
   %%\item \ldots
%%\end{itemize}%%

\section{\IfLanguageName{dutch}{Probleemstelling}{Problem Statement}}
\label{sec:probleemstelling}
CI/CD Procedures zijn zeker geen nieuw idee. Beschrijving over wat deze procedures zijn en hoe ze het best worden opgesteld zijn zeer goed uitgediept. CI/CD wordt gebruikt om snel kwalitatieve software op te leveren en uit te rollen naar productie omgevingen bij klanten. Ook Aucxis wil zich meer toeleggen op beter en kwalitatieve software. Zodanig dat er sneller kwalitatieve support kan worden aangeboden naar de klanten toe. 

Echter, hoe deze nu het best worden geïmplementeerd op zowel praktisch als technisch vlak, bestaat er nog geen eenduidig algemeen antwoord. De meeste recente onderzoeken beschrijven vaak specifieke gevallen en gebruiken vaak compleet op maat gemaakte oplossingen per cloud aanbieder. Dit is ook niet zo onlogisch. Onderzoeken over welk cloud platform optimaal is voor CI/CD pijpleidingen zijn vrijwel niet te vinden. Ook zijn er weinig tot geen onderzoeken te vinden over de vergelijking van verschillende aanbiedingen van cloud aanbieders. Er bestaan wel talloze onderzoeken over de prestatie van de verschillende platformen. Zo een soort onderzoeken zullen ook in acht worden genomen in deze paper.

Deze paper tracht een duidelijk beeld te vormen voor Aucxis over welke aanbiedingen er nu bestaan, welk platform specifieke CI/CD aanbiedingen heeft en wat er nu het meest gebruiksvriendelijk lijkt.

%%Uit je probleemstelling moet duidelijk zijn dat je onderzoek een meerwaarde heeft voor een concrete doelgroep. De doelgroep moet goed gedefinieerd en afgelijnd zijn. Doelgroepen als ``bedrijven,'' ``KMO's,'' systeembeheerders, enz.~zijn nog te vaag. Als je een lijstje kan maken van de personen/organisaties die een meerwaarde zullen vinden in deze bachelorproef (dit is eigenlijk je steekproefkader), dan is dat een indicatie dat de doelgroep goed gedefinieerd is. Dit kan een enkel bedrijf zijn of zelfs één persoon (je co-promotor/opdrachtgever).%%

\section{\IfLanguageName{dutch}{Onderzoeksvraag}{Research question}}
\label{sec:onderzoeksvraag}
Welke Cloud platform aanbieder heeft het meest geschikte aanbod voor CI/CD pijpleidingen te implementeren, vertrekkend vanuit een specifieke use case van Aucxis, naast het Azure Devops platform. Dit zonder arbeidsintensieve containers te maken voor iedere specifieke aanbieder. Dus gebruikmakend van zo veel mogelijk bestaande functie van de cloud platformen. De vergelijkingen zullen vooral gebeuren op aanbod, gebruiksvriendelijkheid, haalbaarheid, prijs en prestatie vergelijkingen voor zover mogelijk.

%%Wees zo concreet mogelijk bij het formuleren van je onderzoeksvraag. Een onderzoeksvraag is trouwens iets waar nog niemand op dit moment een antwoord heeft (voor zover je kan nagaan). Het opzoeken van bestaande informatie (bv. ``welke tools bestaan er voor deze toepassing?'') is dus geen onderzoeksvraag. Je kan de onderzoeksvraag verder specifiëren in deelvragen. Bv.~als je onderzoek gaat over performantiemetingen, dan%%

\section{\IfLanguageName{dutch}{Onderzoeksdoelstelling}{Research objective}}
\label{sec:onderzoeksdoelstelling}
Welke Cloud platform aanbieder nu mogelijk een beter alternatief is voor CI/CD pijpleidingen zal trachten aangetoond te worden in twee stadia. In een eerste fase worden alle kandidaten naast elkaar gelegd voor vergelijking. Er zal worden gekeken naar wat hun aanbiedingen zijn, hoe de prijzen stelsels werken, eerdere implementaties van andere toepassingen, de prestatie van deze platformen en de gebruiksvriendelijkheid. Daaruit wordt dan het beste alternatief voor Azure Devops gekozen. In de tweede fase zal met Azure Devops en het beste alternatief een proof of concept uitgewerkt worden waar dan een aantal testen worden op uitgevoerd. Dit alles zou een vergelijkend beeld moeten vormen over welk platform het meest geschikt is om met bestaande functies een pijpleiding te implementeren.

%%Wat is het beoogde resultaat van je bachelorproef? Wat zijn de criteria voor succes? Beschrijf die zo concreet mogelijk. Gaat het bv. om een proof-of-concept, een prototype, een verslag met aanbevelingen, een vergelijkende studie, enz.%%

\section{\IfLanguageName{dutch}{Opzet van deze bachelorproef}{Structure of this bachelor thesis}}
\label{sec:opzet-bachelorproef}

% Het is gebruikelijk aan het einde van de inleiding een overzicht te
% geven van de opbouw van de rest van de tekst. Deze sectie bevat al een aanzet
% die je kan aanvullen/aanpassen in functie van je eigen tekst.

De rest van deze bachelorproef is als volgt opgebouwd:

In Hoofdstuk~\ref{ch:stand-van-zaken} wordt een overzicht gegeven van de stand van zaken binnen het onderzoeksdomein, op basis van een literatuurstudie.

In Hoofdstuk~\ref{ch:methodologie} wordt de methodologie toegelicht en worden de gebruikte onderzoekstechnieken besproken om een antwoord te kunnen formuleren op de onderzoeksvragen.

% TODO: Vul hier aan voor je eigen hoofstukken, één of twee zinnen per hoofdstuk

In Hoofdstuk~\ref{ch:conclusie}, tenslotte, wordt de conclusie gegeven en een antwoord geformuleerd op de onderzoeksvragen. Daarbij wordt ook een aanzet gegeven voor toekomstig onderzoek binnen dit domein.

%Aantal woorden: 727%