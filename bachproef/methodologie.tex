%%=============================================================================
%% Methodologie
%%=============================================================================

\chapter{\IfLanguageName{dutch}{Methodologie}{Methodology}}
\label{ch:methodologie}
In dit hoofdstuk wordt besproken hoe dit onderzoek zal verlopen. Wat de criteria zijn die beslissend zijn om een kandidaat te selecteren. Wat het uiteindelijke doel is van de 'Proof Of Concep't met deze kandidaat. Wat de criteria zijn voor een succesvolle 'Proof Of Concept'.

In een eerste fase moet er een alternatief Cloud platform worden geselecteerd voor de 'Proof Of Concept'. Deze selectie gebeurt op drie manieren. Eerst wordt er per Cloud platform bekeken wat de algemene mogelijkheden zijn op het platform. Er wordt gekeken naar de standaard mogelijkheden op ieder Cloud platform zoals: virtuele machines, opslag in de Cloud, enz. Dit wordt gedaan om een beeld te kunnen vormen over: hoe complex het platform is, hoe groot het Cloud platform is en hoe vergelijkt het platform in het algemeen. Hierna worden de mogelijkheden voor het versiebeheersysteem besproken. Dit kan belangrijk zijn. Als een Cloud platform de meest populaire systemen niet ondersteund, zou het in de toekomst problemen kunnen geven. Ook zoekt dit onderzoek naar een flexibel platform. Tot slot worden alle mogelijkheden voor CI/CD besproken en vergeleken. Dit is de belangrijkste criteria voor het selecteren van een kandidaat. Als een kandidaat geen ingebouwde oplossingen heeft, dan valt deze buiten beschouwing.

In een tweede fase wordt er bij de kandidaten die in aanmerking komen, een test op gebruiksvriendelijkheid uitgevoerd. Voor deze testen worden er handleidingen gevolgd van de Cloud platformen zelf om een Java applicatie te compileren. Het meten van gebruiksvriendelijkheid gebeurt door middel van tijd. Hoe meer tijd nodig was om op een platform een pipeline te configureren hoe minder gebruiksvriendelijk het platform is. Ook zijn deze testen pas succesvol wanneer de applicatie op een lokale computer kan worden uitgevoerd en de meegeleverde testen zijn uitgevoerd. Ook wordt bij ieder platform opnieuw begonnen met het schrijven van de applicatie. Deze applicatie is hetzelfde voor alle testen.

In de laatste fase wordt een 'Proof Of Concept' gemaakt met de geselecteerde kandidaat. Deze opstelling moet een .Net console applicatie compileren en testen. Hierna moet deze applicatie naar een lokale omgeving worden gekopieerd. Deze applicatie moet dan worden uitgevoerd. Deze opstelling is pas volledig wanneer deze functionaliteit bereikt is. Deze 'Proof Of Concept' wrodt ook nagemaakt in Azure DeVops om een duidelijke conclusie te kunnen maken. Hiervoor zal rekening worden gehouden met de voorgaande fases.

%% TODO: Hoe ben je te werk gegaan? Verdeel je onderzoek in grote fasen, en
%% licht in elke fase toe welke stappen je gevolgd hebt. Verantwoord waarom je
%% op deze manier te werk gegaan bent. Je moet kunnen aantonen dat je de best
%% mogelijke manier toegepast hebt om een antwoord te vinden op de
%% onderzoeksvraag.



%Aantal woorden: 3933%