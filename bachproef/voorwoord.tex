%%=============================================================================
%% Voorwoord
%%=============================================================================

\chapter*{\IfLanguageName{dutch}{Woord vooraf}{Preface}}
\label{ch:voorwoord}
In de eerste plaats wil ik Aucxis bedanken voor het aanbieden van de use case en voor mij de mogelijkheid te bieden om hierover een bachelorproef te schrijven. Ik wil in het bijzonder Raf Boon bedanken voor zijn tijd en uitleg over de werking binnen het bedrijf. Ook wil ik Olivier Roseel bedanken voor de begeleiding van mijn bachelorproef.

Ook wil ik mijn dank uitdrukken naar de gezondheidszorg voor hun werk tijdens de Corona crisis. Dit onderzoek is grotendeels geschreven tijdens de lockdown. Dit heeft mij een bijzonder gevoel gegeven tijdens het schrijven van dit onderzoek. Het was een bijzondere situatie.

Initieel was het niet de bedoeling om zo diep op CI/CD in te gaan. Ik wou vooral aan de slag gaan met Cloud platformen aangezien we hier weinig aanraking mee hadden tijdens onze studie. Dit onderwerp is er gekomen door een voorstel dat ik naar Aucxis heb gedaan. Dit onderwerp hebben we dan in samenspraak preciezer gemaakt zodanig dat het onderwerp niet te ruim werd. Devops procedures interesseren mij enorm. Naar mijn mening is Devops een van de belangrijkste verzameling procedures in de IT-sector. Zonder deze procedures zouden projecten veel meer tijd en geld kosten. Ik had ook nooit gedacht dat CI/CD zo een populair thema is binnen de IT-sector. Ook ben ik geschrokken van hoe weinig onderzoek er is uitgevoerd naar de services en producten voor CI/CD op Cloud platformen. Er bestaan talloze studies over de prestatie van de platformen en hoe deze zijn opgebouwd. Ook bestaan er veel case studies. Echt onderzoek naar een algemeen antwoord op welk Cloud platform er nu het best mogelijkheden voor CI/CD heeft, is moeilijk te vinden. Met dit onderzoek hoop ik de sector helpen een antwoord te bieden aan dit probleem.  Hopelijk komt de ervaring van dit onderzoek van pas op een latere datum.

%% TODO:
%% Het voorwoord is het enige deel van de bachelorproef waar je vanuit je
%% eigen standpunt (``ik-vorm'') mag schrijven. Je kan hier bv. motiveren
%% waarom jij het onderwerp wil bespreken.
%% Vergeet ook niet te bedanken wie je geholpen/gesteund/... heeft

